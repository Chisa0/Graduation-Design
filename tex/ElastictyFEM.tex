\documentclass[a4paper,UTF8,titlepage]{ctexart}
\CTEXsetup[format={\Large\bfseries}]{section}
\usepackage{amsmath}
\usepackage{caption}
\usepackage{graphicx}
\usepackage{float}
%\usepackage{epsfig}
%\usepackage{subfig}
\usepackage{subfigure}
%\usepackage{subcaption}
\usepackage{diagbox}
\pagestyle{plain}

\begin{document}
\title{带间断系数的弹性问题}
\date{\today}
\maketitle

%\renewcommand{\abstractname}{\vspace{-2em}\large\bf 摘要}
%\begin{abstract}
%	摘要 \\
%	\noindent{关键字:\textbf{弹性问题}}
%\end{abstract}
%\thispagestyle{empty}

\tableofcontents

\newpage

\section{引言}

\subsection{研究背景}

具有间断系数的方程成为交界面问题。这类问题起源于许多应用领域,例如2种不同材料或者相同材料在不同状态下物理机制的研究 \cite{邵文婷2017求解一类交界面问题的模态基函数谱元法数值实验}

平面弹性力学方程组是弹性力学中最基础、最常见的模型。当研究的弹性体形状和受力具有一定特点时,通过适当的简化处理,就可以归结为平面弹性问题 \cite{王兆清2018不可压缩平面问题的位移}。

对于各向同性均匀介质的平面弹性问题,当材料的Lam$\acute{e}$常数$\lambda \to \infty$时,即对于几乎不可压介质,通常低阶的协调有限元解,往往不再收敛到原问题的解,或者达不到最优收敛阶,这就是闭锁现象\cite{陈绍春2007平面弹性的一个新的}。

为了消除(近)不可压缩弹性问题中遇到的锁定现象,国内外研究学者提出了多种有效的数值分析方法。根据型函数建立过程中是否需要网格剖分,这些数值方法可以分为两类:一类是有网格方法,这其中包括
高阶有限元法\cite{peet2014legendre}、
混合有限元法\cite{masud2011variational}、
增强有限元法\cite{auricchio2005analysis}和
不连续Galerkin法\cite{hansbo2003discontinuous}等;
另一类是无网格方法,无网格方法又可分为弱形式无网格法和强形式的无网格方法\cite{王兆清2018不可压缩平面问题的位移}。

\subsection{主要研究内容}

本文将通过数值实验的方法,考察对于具有间断系数的平面弹性问题,使用C-R元是否仍可以解除闭锁现象。

\section{有限元近似和闭锁}

\subsection{误差估计}

假设$\Omega$是一个凸多边形区域,并且$\Gamma_1$ or $\Gamma_2$中任意一个为空。对于纯位移问题($\Gamma_2=\emptyset$),我们只需考虑齐次边界条件。
\par
令$\emph{T}^h$ 是$\Omega$ 三角划分的一个非退化族。对于纯位移问题($\Gamma_2=\emptyset$),我们使用有限元空间
% \begin{equation}\label{equ1} (\ref{newton})引用
	%	V_h := \{ \nu \in H^1(\Omega) : \nu |_{\emph{T}} \mbox{为线性函数}, \forall \emph{T} \in \emph{T}^h \}
	% \end{equation}
\\ \\
(1.3.1)
$
\quad \quad \quad
V_h := \{ \nu \in H^1(\Omega) : \nu |_{\emph{T}} \mbox{为线性函数}, \forall \emph{T} \in \emph{T}^h \},
$ 
\\ \\
并且对于纯牵引力问题($\Gamma_1 = \emptyset$),我们使用
% \begin{equation}\label{equ2}
	%	V_h :=
	%\end{equation}
	\\ \\
	(1.3.2)
	$
	\quad \quad \quad
	V_h := \{ \nu \in H^1(\Omega) : \nu |_{\emph{T}} \mbox{为线性函数}, \forall \emph{T} \in \emph{T}^h \},
	$
%	\\ \\
%	根据第二章和第四章的理论我们得到以下定理。
	\\ \\
	\textbf{(1.3.3) Theorem\cite{brenner2008mathematical}.} 令 $u \in H^2(\Omega) \cap H^1(\Omega)$ 满足纯位移问题,并且$u_h \in V_h$ 满足
	$$
	a(u_h, \nu) = \int_{\Omega} f \cdot \nu dx \quad \forall \nu \in V_h.
	$$
	\\
	则存在一个正常数$C_{(\mu, \lambda)}$ 使得
	\\ \\
	(1.3.4)
	$
	\quad \quad \quad
	\| u - u_h \|_{H^1(\Omega)} \le C_{(\mu, \lambda)} h \| u \|_{H^2(\Omega)}.
	$
	\\ \\
	\textbf{(1.3.5) Theorem.} 令 $u \in H^2(\Omega)$ 满足纯牵引力问题。令 $u_h \in V_h$ 满足
	$$
	a(u_h,\nu) = \int_{\Omega} f \cdot \nu dx + \int_{\Gamma} t \cdot \nu ds \quad \forall \nu \in V_h.
	$$ 
	\\
	则存在一个正常数$C_{(\mu, \lambda)}$ 使得
	$$
	\| u - u_h \|_{H^1(\Omega)} \le C_{(\mu, \lambda)} h \| u \|_{H^2(\Omega)}.
	$$
%	对于一般情况 $ \emptyset \ne \Gamma_1 \ne \partial \Omega $ 下的收敛定理,查看练习 11.x.25. 
%	\\

\subsection{闭锁现象}

	对于固定的 $\mu$ 和 $\lambda$,定理 1.3.3 和 1.3.5 给出了弹性问题令人满意近似的有限元近似。但是这些有限元方法的性能随着 $\lambda$ 趋向于 $\infty$ 而变差。这就是所谓的锁定现象。
	\par 
	令 $\Omega = (0,1) \times (0,1)$. 我们考虑 $\mu = 1$ 时的纯位移边值问题:
	\\ \\
	(1.3.6)
	$
	\quad \quad \quad
	\begin{matrix}
		div \{ 2 \epsilon (u^{\lambda}) + \lambda tr (\epsilon (u^{\lambda})) \delta \} = f \quad in \quad \Omega \\ \quad \quad \quad \quad \quad 
		u^{\lambda}|_{\partial \Omega} =  0.
	\end{matrix}
	$
	\\
	注意给定的 f ,当 $\lambda \to \infty$,(1.2.33)说明 $\| div u^{\lambda} \|_{H^1(\Omega)} \to 0$.换句话说,我们正在处理一种几乎不可能压缩的弹性材料。为了强调对 $\lambda$ 的依赖,我们将应力张量(1.1.3) $\sigma_{\lambda}(\nu)$ 和 变分形式(1.2.2) $a_{\lambda}(\nu,\omega)$ 表示为
	$$
	\begin{matrix}
		\sigma_{\lambda}(\nu) = 2 \epsilon(\nu) + \lambda tr (\epsilon(\nu)) \delta \\
		a_{\lambda}(\nu,\omega) = \int_{\Omega} \{ 2 \epsilon(\nu) : \epsilon(\omega) + \lambda div \nu div \omega \} dx.
	\end{matrix}
	$$
	
	令 $\emph{T}^h$ 为 $\Omega $ (图 1) 的一个规则三角剖分,并且 $V_h$ 被定义为 (1.3.1)。对于每一个 $u \in H^2(\Omega) \cap H_0^1(\Omega)$,定义 $u_h^{\lambda} \in V_h$ 为以下方程组的特解
	$$
	a_{\lambda}(u_h^{\lambda},\nu) = \int_{\Omega} 
	[ -div \sigma_{\lambda}(u) ] \cdot \nu dx \quad \forall \nu \in V_h.
	$$
	
	\begin{figure}[hbt]
		\centering
		\includegraphics{../image/Fig.11.1.png}
		\caption{单位正方形的规则三角剖分}
	\end{figure}
	
	定义 $L_{\lambda,h}$ 为
	$$
	L_{\lambda,h} := sup \{ \frac{|u-u_h^{\lambda}|_{H^1(\Omega)}}{\| div \sigma_{\lambda}(u) \|_{L^2(\Omega)}} : 0 \ne u \in H^2(\Omega) \cap H^1(\Omega) \}.
	$$
	\\ 
	我们要证明存在一个与 h 无关的正常数 C 使得
	\\ \\
	(1.3.7)
	$
	\quad \quad \quad \quad \quad \quad \quad \quad
	\lim\limits_{\lambda \to \infty} \inf L_{\lambda,h} \ge C.
	$
	\\ \\
	式 (1.3.7) 意味着:无论 h 取多小,只要 $\lambda$ 足够大,我们都能找到 $u \in H^2(\Omega) \cap H^1(\Omega)$ 使得相对误差$ |u-u_h|_{H^1(\Omega)} / \| div \sigma_{\lambda}(u) \|_{L^2(\Omega)} $ 以一个与 h 无关的常数为下界。换句话说,有限元方法的性能将会随着 $\lambda$ 变大而变坏。
	\par
	为证明式 (1.3.7),我们首先观察到
	\\ \\
	(1.3.8)
	$
	\quad \quad \quad \quad \quad \quad \quad
	\{ \nu \in V_h : div \nu = 0 \} = \{ 0 \}
	$
	\\ \\
	(cf.exercise 11.x.14). 因此,映射 $\nu \to div \nu$ 是有限维空间 $V_h$ 到 $L^2(\Omega)$ 的一个一对一映射,并且存在一个正常数 $C_1(h)$ 使得
	\\ \\
	(1.3.9)
	$
	\quad \quad \quad \quad \quad
	\| \nu \|_{H^1(\Omega)} \le C_1(h) \| div \nu \|_{L^2(\Omega)} \quad \forall \nu \in V_h.
	$
	\\ \par 
	令 $\psi$ 是 $\overline{\Omega}$ 上的无穷次可微函数,使得在 $\Omega$ 的边界上 $curl \psi = 0$ 且 $\| \epsilon(curl \psi) \|_{L^2(\Omega)} = 1$。令 $u := curl \psi$。则 $u \in H^2(\Omega) \cap H^1(\Omega)$,并有
	\\ \\
	(1.3.10)
	$
	\quad \quad \quad \quad \quad \quad \quad \quad \quad \quad \quad
	div u = 0,
	$
	\\
	(1.3.11)
	$
	\quad \quad \quad \quad \quad \quad \quad \quad
	\| \epsilon(u) \|_{L^2(\Omega)} = 1,
	$
	\\
	(1.3.12)
	$
	\quad \quad \quad \quad \quad \quad \quad \quad \quad \quad 
	\sigma_{\lambda}(u) = 2 \epsilon(u).
	$
	\\ \par 
	根据 (1.3.10),(1.3.11) 和 1.2 节开始的分步积分得
	\\ \\
	(1.3.13)
	$
	\quad \quad \quad \quad
	- \int_{\Omega} div \epsilon(u) \cdot u dx = \int_{\Omega} \epsilon(u) : \epsilon(u) dx = 1.
	$
	\\ \\
	根据 (1.3.12),(1.3.13) 推断
	\\ \\
	(1.3.14)
	$
	\quad \quad \quad \quad \quad \quad \quad
	\lim\limits_{\lambda \to \infty} div \sigma_{\lambda}(u) = 2 div \epsilon(u) \ne 0. 
	$
	\\ \\
	由 (2.5.10) 得,
	\\ \\
	(1.3.15)
	$
	\quad \quad \quad 
	a_{\lambda}(u-u_h^{\lambda}, u-u_h^{\lambda}) = \min\limits_{\nu \in V_h} a_{\lambda}(u-\nu,u-\nu) \le a_{\lambda}(u,u).
	$
	\\ \\
	由 (1.3.10) 和 (1.3.11),我们得到
	\\ \\
	(1.3.16)
	$
	\quad \quad \quad \quad \quad \quad \quad \quad \quad
	a_{\lambda}(u,u) = 2.
	$
	\\ \\
	因此,对于 $\lambda$ 足够大时有
	\\ \\
	(1.3.17)
	$
	\quad \quad \quad \quad \quad \quad \quad 
	a_{\lambda} (u-u_h^{\lambda}, u-u_h^{\lambda}) \le 2.
	$
	\\ \\ 
	由 (1.3.10) 和 (1.3.17) 得
	\\
	$$
	\begin{matrix}
		\sqrt{\lambda} \| div u_h^{\lambda} \|_{L^2(\Omega)} = \sqrt{\lambda} \| div(u-u_h^{\lambda}) \|_{L^2(\Omega)} \\ 
		\quad \quad \quad \quad \quad \quad \quad
		\le \sqrt{a_{\lambda}(u-u_h^{\lambda}, u-u_h^{\lambda})} \\
		\le \sqrt{2}	
	\end{matrix}
	$$
	\\
	对足够大的 $\lambda$ 有
	$$
	\lim\limits_{\lambda \to \infty} \| div u_h^{\lambda} \|_{L^2(\Omega)} = 0.
	$$
	\\
	由式 (1.3.9) 有
	\\ \\
	(1.3.18)
	$
	\quad \quad \quad \quad \quad \quad \quad \quad 
	\lim\limits_{\lambda \to \infty} \| u_h^{\lambda} \|_{H^1(\Omega)} = 0.
	$
	\\ \\
	最后,我们得到 %(cf.exercise 11.x.16)
	\\ \\
	(1.3.19) 
	$
	\quad \quad \quad \quad \quad 
	\begin{matrix}
		\lim\limits_{\lambda \to \infty}\inf L_{\lambda,h} \ge \lim\limits_{\lambda \to \infty}\inf \frac{|u-u_h^{\lambda}|_{H^1(\Omega)}}{\| div \sigma_{\lambda}(u) \|_{L^2(\Omega)}} \\
		\quad \quad \quad \quad
		= \frac{|u|_{H^1(\Omega)}}{\| div \sigma(u) \|_{L^2(\Omega)}} > 0.
	\end{matrix}
	$

\subsection{剖分与基函数}

\subsubsection{剖分}

对区间 $\Omega$ 按图1方式剖分,并对节点和单元进行编号,各节点坐标为$(x_i,y_i)$, i=0, ... , n,

\begin{figure}[hbt]
	\centering
	\includegraphics[height=4cm,width=4cm]{../image/subdivsion.png}
	\caption{}
\end{figure}

设基函数为 
$$ 
(\varphi_0,0)^t, (0,\varphi_0)^t, (\varphi_1,0)^t, (0,\varphi_1)^t, ... , (\varphi_n,0)^t, (0,\varphi_n)^t 
$$

$\varphi_i$ 为线性元,以下得到其在各单元上表达式。 

\subsubsection{线性元}

如图2,设 $ \bigtriangleup(p_0,p_1,p_2) $ 是以 $p_0,p_1,p_2$ 为顶点的任意三角型元,面积为S。在 $ \bigtriangleup (p_0,p_1,p_2) $ 内任取一点$p_3$,坐标为$(x,y)$。过$p_3$点作与三个顶点的连线,将 $ \bigtriangleup(p_0,p_1,p_2) $ 分成三个三角形: $ \bigtriangleup(p_1,p_2,p_3), \bigtriangleup(p_0,p_3,p_2), \bigtriangleup(p_0,p_1,p_3) $,其面积分别为$S_0,S_1,S_2$

\begin{figure}[hbt]
	\centering
	\includegraphics[height=3cm,width=4cm]{../image/TriangleElement.png}
	\caption{}
	\label{SampleOfDatasets}
\end{figure}

显然$S_0 + S_1 + S_2 = S$,令
$$
L_0 = \frac{S_0}{S}, \quad L_1 = \frac{S_1}{S}, \quad L_2 = \frac{S_2}{S}
$$
\par
%称$(L_0,L_1,L_2)$位$P_3$的面积坐标,其中
%$$
%	\begin{cases}
	%		2S = \left| \begin{matrix}
		%				1 & x_0 & y_0 \\
		%				1 & x_1 & y_1 \\
		%				1 & x_2 & y_2
		%			 \end{matrix} \right| ,
	%		 \quad
	%		 2S_0 = \left| \begin{matrix}
		%		 			1 & x   & y   \\
		%		 			1 & x_1 & y_1 \\
		%		 			1 & x_2 & y_2
		%		 \end{matrix} \right| 
	%		 \\
	%		2S_1 = \left| \begin{matrix}
		%					1 & x_0 & y_0 \\
		%					1 & x   & y   \\
		%					1 & x_2 & y_2
		%			   \end{matrix} \right|,
	%		\quad
	%		2S_2 = \left| \begin{matrix}
		%					1 & x_0 & y_0 \\
		%					1 & x_1 & y_1 \\
		%					1 & x   & y
		%			   \end{matrix} \right|
	%	\end{cases}
%$$

%由此可得面积坐标和直角坐标的转化关系
%$$
%\begin{cases}
%	x = x_0 L_0 + x_1 L_1 + x_2 L_2 \\
%	y = y_0 L_0 + y_1 L_1 + x_2 L_2
%\end{cases}
%$$
$$
\begin{cases}
	L_0 = \frac{1}{2S} [(x_2 y_3 - x_3 y_2) + (y_2 - y_3) x + (x_3 - x_2) y] \\
	L_1 = \frac{1}{2S} [(x_3 y_0 - x_0 y_3) + (y_3 - y_0) x + (x_0 - x_3) y] \\
	L_2 = \frac{1}{2S} [(x_0 y_1 - x_1 y_0) + (y_0 - y_1) x + (x_1 - x_0) y]
\end{cases} 
$$

因为

$$
\begin{cases}
	L_0 = \begin{cases}
		1, \quad x = x_0, y = y_0 \\
		0, \quad x = x_1, y = y_1 \\
		0, \quad x = x_2, y = y_2
	\end{cases} \\
	L_1 = \begin{cases}
		0, \quad x = x_0, y = y_0 \\
		1, \quad x = x_1, y = y_1 \\
		0, \quad x = x_2, y = y_2
	\end{cases} \\
	L_2 = \begin{cases}
		0, \quad x = x_0, y = y_0 \\
		0, \quad x = x_1, y = y_1 \\
		1, \quad x = x_2, y = y_2
	\end{cases} \\
\end{cases}
$$

所以在此区间上 $\varphi_i = L_i$。


\section{算例}

\subsection{弹性问题}

\subsubsection{模型}

令
$$
\begin{matrix}
	\sigma(u) = 2 \mu \epsilon(u) + \lambda tr(\epsilon(u)) \delta \\
	\epsilon(u) = \frac{1}{2} (grad u + (grad u)^t) \\
	tr(\tau) = \tau_{11} + \tau_{22} \\
	grad(u) = \begin{pmatrix}
		\frac{\partial u_1}{\partial x} & \frac{\partial u_1}{\partial y} \\
		\frac{\partial u_2}{\partial x} &
		\frac{\partial u_2}{\partial y}
	\end{pmatrix} \\
	\delta = \begin{pmatrix}
		1 & 0 \\
		0 & 1
	\end{pmatrix} \\
	div u = \frac{\partial u_1}{\partial x} + \frac{\partial u_2}{\partial y} \\
	div \tau = \begin{pmatrix}
		\frac{\partial \tau_{11}}{\partial x} + \frac{\partial \tau_{12}}{\partial y} \\
		\frac{\partial \tau_{12}}{\partial x} + \frac{\partial \tau_{22}}{\partial 
		y}
	\end{pmatrix}
\end{matrix}
$$

考察模型
$$
\begin{matrix}
	-div \sigma(u) = f \quad \in \Omega  \\
	u |_{\Gamma} = 0
	%(\sigma(u) \nu) |_{\Gamma2} = t
\end{matrix}
$$ 
\par
其中$ u = (u_1,u_2)^t $ 为求解向量,$ f = (f_1,f_2)^t $为右端向量,$ \Omega = [0,1] \times [0,1] $
$$
\begin{matrix}
	u_1 = (x - 1)(y - 1) y sin(x) 
	\\
	u_2 = (x - 1)(y - 1) x sin(y) 
	\\
	f_1 = -((2 \mu + \lambda) y (y - 1) (2 cos(x) - (x - 1) sin(x)) \\
	+ (\mu + \lambda) (2 x - 1) (sin(y) + (y - 1) cos(y)) \\
	+ 2 \mu (x - 1) sin(x)) 
	\\
	f_2 = -((2 \mu + \lambda) x (x - 1) (2 cos(y) - (y - 1) sin(y)) \\
	+ (\mu + \lambda) (2 y - 1) (sin(x) + (x - 1) cos(x)) \\ 
	+ 2 \mu (y - 1) sin(y))
\end{matrix}
$$

\subsubsection{变分}

设 $\nu = (\nu_1,\nu_2)^t, \quad \nu_1, \nu_2 \in C_0^{\infty}(\Omega)$,方程两边同乘 $\nu$ 并积分得
$$
-\int_{\Omega} div \sigma(u) \nu dxdy = \int_{\Omega} f \nu dxdy
$$

由
$$
\begin{matrix}
	f div a = div(fa) - a : grad f \\
	\int_{\Omega} div a dV = \int_{\partial \Omega} a dS
\end{matrix}
$$

得
\par \quad \quad
$-\int_{\Omega} div \sigma(u) \nu dxdy$
$$ 
\quad \quad
\begin{matrix}
	\begin{aligned}
		&= -\int_{\Omega} div(\sigma(u) \nu) dxdy - \int_{\Omega} \sigma(u) : grad \nu dxdy \\
		&= -\int_{\Gamma} \sigma(u) \nu dxdy + \int_{\Omega} \sigma(u) : grad \nu dxdy \\
		&= \int_{\Omega} \sigma(u) : grad \nu dxdy 
	\end{aligned}
\end{matrix}
$$

所以
$$
\int_{\Omega} \sigma(u) : grad \nu dxdy = \int_{\Omega} f \nu dxdy
$$

该问题的变分问题为,求$u \in H^1(\Omega)$ 使得 $u |_{\Gamma_1} = g$,并且
$$
	a(u,\nu) = \int_{\Omega} f \cdot \nu dxdy \quad \forall \nu \in V
$$ 
\par
其中
$$
	\begin{matrix}
		\begin{aligned}
			a(u,\nu) :&= \int_{\Omega} \sigma(u) : grad \nu dxdy \\  
			&= \int_{\Omega} 2 \mu \epsilon(u) : grad \nu + \lambda div u div \nu dxdy \\
			V :&= \{ \nu \in H^1(\Omega) \quad | \quad \nu |_{\Gamma} = 0 \}
		\end{aligned}
	\end{matrix}
$$

%证其与原问题的等价性

%\begin{enumerate}
%	\item 若 u 为原问题的解 \\

	
%	\item 若 u 为变分问题的解 \\
%	由
%	\par \quad \quad
%	$\int_{\Omega} \sigma(u) : grad \nu dxdy$
%	$$
%	\quad \quad
%	\begin{matrix}
%		\begin{aligned}
%			&= -\int_{\Gamma} \sigma(u) \nu dxdy + \int_{\Omega} \sigma(u) : grad %\nu dxdy \\
%			&= -\int_{\Omega} div \sigma(u) \nu dxdy
%		\end{aligned}
%	\end{matrix} 
%	$$
%
%	得
%	$$
%		-\int_{\Omega} div \sigma(u) \nu dxdy = \int_{\Omega} f \nu dxdy
%	$$
%	
%	由变分法基本引理得
%	$$
%		-div \sigma(u) = f
%	$$
%	
%\end{enumerate}

\subsubsection{刚度矩阵}

设 $\varphi_{xi} = (\varphi_i, 0)^t, \varphi_{yi} = (0,\varphi_i)^t$,i = 0, ... , n 为试探函数空间$U_h$的基函数,则任一 $u_h \in U_h$ 可表成
$$
	u_h = \sum\limits_{i=1}^n u^{i}_1 \varphi_{xi} + \sum\limits_{i=1}^n u^{i}_2 \varphi_{yi}, \quad u^i = u_h(x_i,y_i)
$$ 
\\ 
令 $\phi_{2i} = \varphi_{xi}$ ,$\phi_{2i+1} = \varphi_{yi}$ , $c_{2i} = u^i_1$ , $c_{2i+1} = u^i_2$
带入变分形式得
$$
	\sum\limits_{j=0}^{2n+1} a(\phi_j, \phi_i) c_i = (f,\phi_i) \quad i=0, ... ,2n+1
$$

矩阵形式为
$$
\begin{matrix}
	A c = F \\
	A = (a(\phi_i, \phi_j))_{(2n+1) \times (2n+1)} \\
	F = ((f,\phi_i))_{(2n+1) \times 1} \\
	c = (c_i)_{(2n+1) \times 1}
\end{matrix}
$$

%\subsection{单元刚度矩阵}
%
%在第m个单元cell=$\bigtriangleup(i,j,k)$上,单元刚度矩阵和单元载荷向量为
%$$
%\begin{matrix}
%	A^{(m)} = (\int_{cell} (2 \mu \epsilon(\phi_{i_1}) : \epsilon(\phi_{j_1}) + %\lambda div \phi_{i_1} div \phi_{j_1}))_{3 \times 3} \\
%	F^{(m)} = (\int_{cell} f \cdot \phi_{i_1})_{3 \times 1} \\
%	i_1, j_1 = i, j, k
%\end{matrix} 
%$$
%
%将$A^{(m)}$扩展成$n \times n$矩阵,行列为i,j,k的九个元素即为$A^{(m)}$的九个元素,并以同样的方式将$F^{(m)}$扩展成$n \times 1$向量,则
%$$
%\begin{matrix} 
%	A = \sum\limits_{m=0}^n A^{(m)} \\
%	F = \sum\limits_{m=0}^n F^{(m)}
%\end{matrix}
%$$

\subsubsection{边界条件}

模型为齐次边界条件,若$(x_i,y_i)$为边界点,则 A 第 2i 行第 2i 列,第 2i+1 行第 2i+1 列元素为1,其他元素及  F(2i),F(2i+1) 都为0。

%\vfill \newpage

%\bibliographystyle{unsrt}
%\bibliography{interface_problem}

\subsubsection{实验结果}

\begin{table}[ht]
	\caption{线性元误差}
	\begin{tabular}{|c|c|c|c|c|c|c|} \hline
		\diagbox{$\lambda$}{h} &3.125e-2 &7.812e-3 &1.953e-3 &4.882e-4 &1.220e-4 &3.051e-5 \\ \hline
		1 &3.1007e-3 &6.6322e-4 &1.4833e-4 &4.3538e-5 &1.6079e-5 &6.2883e-6 \\ \hline
		10 &3.4359e-3 &1.3916e-3 &4.5128e-4 &1.2428e-4 &3.2305e-5 &8.1245e-6 \\ \hline
		100 &3.3958e-3 &9.0716e-3 &3.6454e-3 &1.0038e-3 &2.6076e-4 &6.5941e-5 \\ \hline
		1e3 &3.4040e-3 &8.5781e-2 &3.5580e-2 &9.8154e-3 &2.5446e-3 &6.4444e-4 \\ \hline
		1e4 &3.4050e-3 &8.5287e-1 &3.5492e-1 &9.7931e-2 &2.5383e-2 &6.4295e-3 \\ \hline
		1e5 &3.4051e-3 &8.5237 &3.5484 &9.7909e-1 &2.5377e-1 &6.4280e-2 \\ \hline
	\end{tabular}
\end{table}

\begin{figure}[ht]
	\centering
	\subfigure[$\lambda$=1]{
		\includegraphics[height=2cm,width=3cm]{../image/elasticityLfemLam_1.png}
		\label{1}}	
	\subfigure[$\lambda$=10]{
		\includegraphics[height=2cm,width=3cm]{../image/elasticityLfemLam_10.png}
		\label{10}}
	\subfigure[$\lambda$=100]{
	\includegraphics[height=2cm,width=3cm]{../image/elasticityLfemLam_100.png}
	\label{100}}

 	\subfigure[$\lambda$=1000]{
	\includegraphics[height=2cm,width=3cm]{../image/elasticityLfemLam_1000.png}
	\label{1000}}
	\subfigure[$\lambda$=10000]{
	\includegraphics[height=2cm,width=3cm]{../image/elasticityLfemLam_10000.png}
	\label{10000}}
	\subfigure[$\lambda$=100000]{
	\includegraphics[height=2cm,width=3cm]{../image/elasticityLfemLam_100000.png}
	\label{100000}}
	\caption{线性元误差}
\end{figure}

\subsection{弹性界面问题}

\subsubsection{模型}

考察模型
$$
\begin{matrix}
	-div \sigma(u) = f \quad \in \Omega  \\
	u |_{\Gamma} = 0
	%(\sigma(u) \nu) |_{\Gamma2} = t
\end{matrix}
$$ 
\par
其中$ u = (u_1,u_2)^t $ 为求解向量,$ f = (f_1,f_2)^t $为右端向量,$ \Omega = [0,2] \times [0,1] $

当$(x,y) \in [0,1] \times [0,1]$时,$\mu = \lambda = 1$
$$
\begin{matrix}
	u_1 = (x - 1)(y - 1) y sin(x) 
	\\
	u_2 = (x - 1)(y - 1) x sin(y) 
\end{matrix}
$$

当$(x,y) \in [1,2] \times [0,1]$时, $\mu = \lambda = 2$

$$
\begin{matrix}
	u_1 = (x - 1)(y - 1) y sin(x) 
	\\
	u_2 = (x - 1)(y - 1) x sin(y) 
\end{matrix}
$$

右端向量

$$
\begin{matrix}
	f_1 = -((2 \mu + \lambda) y (y - 1) (2 cos(x) - (x - 1) sin(x)) \\
	+ (\mu + \lambda) (2 x - 1) (sin(y) + (y - 1) cos(y)) \\
	+ 2 \mu (x - 1) sin(x)) 
	\\
	f_2 = -((2 \mu + \lambda) x (x - 1) (2 cos(y) - (y - 1) sin(y)) \\
	+ (\mu + \lambda) (2 y - 1) (sin(x) + (x - 1) cos(x)) \\ 
	+ 2 \mu (y - 1) sin(y))
\end{matrix}
$$

\subsubsection{变分}

设 $\nu = (\nu_1,\nu_2)^t, \quad \nu_1, \nu_2 \in C_0^{\infty}(\Omega)$,方程两边同乘 $\nu$ 并积分得
$$
-\int_{\Omega} div \sigma(u) \nu dxdy = \int_{\Omega} f \nu dxdy
$$

由
$$
\begin{matrix}
	f div a = div(fa) - a : grad f \\
	\int_{\Omega} div a dV = \int_{\partial \Omega} a dS
\end{matrix}
$$

得
\par \quad \quad
$-\int_{\Omega} div \sigma(u) \nu dxdy$
$$ 
\quad \quad
\begin{matrix}
	\begin{aligned}
		&= -\int_{\Omega} div(\sigma(u) \nu) dxdy - \int_{\Omega} \sigma(u) : grad \nu dxdy \\
		&= -\int_{\Gamma} \sigma(u) \nu dxdy + \int_{\Omega} \sigma(u) : grad \nu dxdy \\
		&= \int_{\Omega} \sigma(u) : grad \nu dxdy 
	\end{aligned}
\end{matrix}
$$

所以
$$
\int_{\Omega} \sigma(u) : grad \nu dxdy = \int_{\Omega} f \nu dxdy
$$

该问题的变分问题为,求$u \in H^1(\Omega)$ 使得 $u |_{\Gamma_1} = g$,并且
$$
a(u,\nu) = \int_{\Omega} f \cdot \nu dxdy \quad \forall \nu \in V
$$ 
\par
其中
$$
\begin{matrix}
	\begin{aligned}
		a(u,\nu) :&= \int_{\Omega} \sigma(u) : grad \nu dxdy \\  
		&= \int_{\Omega} 2 \mu \epsilon(u) : grad \nu + \lambda div u div \nu dxdy \\
		V :&= \{ \nu \in H^1(\Omega) \quad | \quad \nu |_{\Gamma} = 0 \}
	\end{aligned}
\end{matrix}
$$

\newpage
\vfill

\bibliographystyle{unsrt}
\bibliography{interface_problem}

\end{document}