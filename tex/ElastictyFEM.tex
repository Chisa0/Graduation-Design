\documentclass[UTF8,titlepage,twocolumn]{ctexart}
\CTEXsetup[format={\Large\bfseries}]{section}
\usepackage{amsmath}
\usepackage{caption}
\usepackage{graphicx}
\pagestyle{plain}
\title{弹性问题}
\date{\today}
\begin{document}
\maketitle

\section{模型}

令
$$
\begin{matrix}
	\sigma(u) = 2 \mu \epsilon(u) + \lambda tr(\epsilon(u)) \delta \\
	\epsilon(u) = \frac{1}{2} (grad u + (grad u)^t) \\
	tr(\tau) = \tau_{11} + \tau_{22} \\
	grad(u) = \begin{pmatrix}
		\frac{\partial u_1}{\partial x} & \frac{\partial u_1}{\partial y} \\
		\frac{\partial u_2}{\partial x} &
		\frac{\partial u_2}{\partial y}
	\end{pmatrix} \\
	\delta = \begin{pmatrix}
		1 & 0 \\
		0 & 1
	\end{pmatrix} \\
	div u = \frac{\partial u_1}{\partial x} + \frac{\partial u_2}{\partial y} \\
	div \tau = \begin{pmatrix}
		\frac{\partial \tau_{11}}{\partial x} + \frac{\partial \tau_{12}}{\partial y} \\
		\frac{\partial \tau_{12}}{\partial x} + \frac{\partial \tau_{22}}{\partial 
		y}
	\end{pmatrix}
\end{matrix}
$$

考察模型
$$
\begin{matrix}
	-div \sigma(u) = f \quad \in \Omega  \\
	u |_{\Gamma} = 0
	%(\sigma(u) \nu) |_{\Gamma2} = t
\end{matrix}
$$ 
\par
其中$ u = (u_1,u_2)^t $ 为求解向量,$ f = (f_1,f_2)^t $为右端向量,$ \Omega = [0,1] \times [0,1] $
$$
\begin{matrix}
	u_1 = (x - 1)(y - 1) y sin(x) 
	\\
	u_2 = (x - 1)(y - 1) x sin(y) 
	\\
	f_1 = -((2 \mu + \lambda) y (y - 1) (2 cos(x) - (x - 1) sin(x)) \\
	+ (\mu + \lambda) (2 x - 1) (sin(y) + (y - 1) cos(y)) \\
	+ 2 \mu (x - 1) sin(x)) 
	\\
	f_2 = -((2 \mu + \lambda) x (x - 1) (2 cos(y) - (y - 1) sin(y)) \\
	+ (\mu + \lambda) (2 y - 1) (sin(x) + (x - 1) cos(x)) \\ 
	+ 2 \mu (y - 1) sin(y))
\end{matrix}
$$

\section{变分}

设 $\nu = (\nu_1,\nu_2)^t, \quad \nu_1, \nu_2 \in C_0^{\infty}(\Omega)$,方程两边同乘 $\nu$ 并积分得
$$
-\int_{\Omega} div \sigma(u) \nu dxdy = \int_{\Omega} f \nu dxdy
$$

由
$$
\begin{matrix}
	f div a = div(fa) - a : grad f \\
	\int_{\Omega} div a dV = \int_{\partial \Omega} a dS
\end{matrix}
$$

得
\par \quad \quad
$-\int_{\Omega} div \sigma(u) \nu dxdy$
$$ 
\quad \quad
\begin{matrix}
	\begin{aligned}
		&= -\int_{\Omega} div(\sigma(u) \nu) dxdy - \int_{\Omega} \sigma(u) : grad \nu dxdy \\
		&= -\int_{\Gamma} \sigma(u) \nu dxdy + \int_{\Omega} \sigma(u) : grad \nu dxdy \\
		&= \int_{\Omega} \sigma(u) : grad \nu dxdy 
	\end{aligned}
\end{matrix}
$$

所以
$$
\int_{\Omega} \sigma(u) : grad \nu dxdy = \int_{\Omega} f \nu dxdy
$$

该问题的变分问题为,求$u \in H^1(\Omega)$ 使得 $u |_{\Gamma_1} = g$,并且
$$
	a(u,\nu) = \int_{\Omega} f \cdot \nu dxdy \quad \forall \nu \in V
$$ 
\par
其中
$$
	\begin{matrix}
		\begin{aligned}
			a(u,\nu) :&= \int_{\Omega} \sigma(u) : grad \nu dxdy \\  
			&= \int_{\Omega} 2 \mu \epsilon(u) : grad \nu + \lambda div u div \nu dxdy \\
			V :&= \{ \nu \in H^1(\Omega) \quad | \quad \nu |_{\Gamma} = 0 \}
		\end{aligned}
	\end{matrix}
$$

%证其与原问题的等价性

%\begin{enumerate}
%	\item 若 u 为原问题的解 \\

	
%	\item 若 u 为变分问题的解 \\
%	由
%	\par \quad \quad
%	$\int_{\Omega} \sigma(u) : grad \nu dxdy$
%	$$
%	\quad \quad
%	\begin{matrix}
%		\begin{aligned}
%			&= -\int_{\Gamma} \sigma(u) \nu dxdy + \int_{\Omega} \sigma(u) : grad %\nu dxdy \\
%			&= -\int_{\Omega} div \sigma(u) \nu dxdy
%		\end{aligned}
%	\end{matrix} 
%	$$
%
%	得
%	$$
%		-\int_{\Omega} div \sigma(u) \nu dxdy = \int_{\Omega} f \nu dxdy
%	$$
%	
%	由变分法基本引理得
%	$$
%		-div \sigma(u) = f
%	$$
%	
%\end{enumerate}

\section{剖分与基函数}

\subsection{剖分}

对区间 $\Omega$ 按图1方式剖分,并对节点和单元进行编号,各节点坐标为$(x_i,y_i)$, i=0, ... , n,

\begin{figure}[hbt]
	\centering
	\includegraphics[height=4cm,width=4cm]{../image/subdivsion.png}
	\caption{}
\end{figure}

设基函数为 
$$ 
(\varphi_0,0)^t, (0,\varphi_0)^t, (\varphi_1,0)^t, (0,\varphi_1)^t, ... , (\varphi_n,0)^t, (0,\varphi_n)^t 
$$

$\varphi_i$ 为线性元,以下得到其在各单元上表达式。 

\subsection{线性元}

如图2,设 $ \bigtriangleup(p_0,p_1,p_2) $ 是以 $p_0,p_1,p_2$ 为顶点的任意三角型元,面积为S。在 $ \bigtriangleup (p_0,p_1,p_2) $ 内任取一点$p_3$,坐标为$(x,y)$。过$p_3$点作与三个顶点的连线,将 $ \bigtriangleup(p_0,p_1,p_2) $ 分成三个三角形: $ \bigtriangleup(p_1,p_2,p_3), \bigtriangleup(p_0,p_3,p_2), \bigtriangleup(p_0,p_1,p_3) $,其面积分别为$S_0,S_1,S_2$

\begin{figure}[hbt]
	\centering
	\includegraphics[height=3cm,width=4cm]{../image/TriangleElement.png}
	\caption{}
	\label{SampleOfDatasets}
\end{figure}

显然$S_0 + S_1 + S_2 = S$,令
$$
	L_0 = \frac{S_0}{S}, \quad L_1 = \frac{S_1}{S}, \quad L_2 = \frac{S_2}{S}
$$
\par
%称$(L_0,L_1,L_2)$位$P_3$的面积坐标,其中
%$$
%	\begin{cases}
%		2S = \left| \begin{matrix}
%				1 & x_0 & y_0 \\
%				1 & x_1 & y_1 \\
%				1 & x_2 & y_2
%			 \end{matrix} \right| ,
%		 \quad
%		 2S_0 = \left| \begin{matrix}
%		 			1 & x   & y   \\
%		 			1 & x_1 & y_1 \\
%		 			1 & x_2 & y_2
%		 \end{matrix} \right| 
%		 \\
%		2S_1 = \left| \begin{matrix}
%					1 & x_0 & y_0 \\
%					1 & x   & y   \\
%					1 & x_2 & y_2
%			   \end{matrix} \right|,
%		\quad
%		2S_2 = \left| \begin{matrix}
%					1 & x_0 & y_0 \\
%					1 & x_1 & y_1 \\
%					1 & x   & y
%			   \end{matrix} \right|
%	\end{cases}
%$$

%由此可得面积坐标和直角坐标的转化关系
%$$
%\begin{cases}
%	x = x_0 L_0 + x_1 L_1 + x_2 L_2 \\
%	y = y_0 L_0 + y_1 L_1 + x_2 L_2
%\end{cases}
%$$
$$
	\begin{cases}
		L_0 = \frac{1}{2S} [(x_2 y_3 - x_3 y_2) + (y_2 - y_3) x + (x_3 - x_2) y] \\
		L_1 = \frac{1}{2S} [(x_3 y_0 - x_0 y_3) + (y_3 - y_0) x + (x_0 - x_3) y] \\
		L_2 = \frac{1}{2S} [(x_0 y_1 - x_1 y_0) + (y_0 - y_1) x + (x_1 - x_0) y]
	\end{cases} 
$$

因为

$$
\begin{cases}
	L_0 = \begin{cases}
			 1, \quad x = x_0, y = y_0 \\
			 0, \quad x = x_1, y = y_1 \\
			 0, \quad x = x_2, y = y_2
		  \end{cases} \\
	L_1 = \begin{cases}
			 0, \quad x = x_0, y = y_0 \\
			 1, \quad x = x_1, y = y_1 \\
			 0, \quad x = x_2, y = y_2
		  \end{cases} \\
	L_2 = \begin{cases}
			 0, \quad x = x_0, y = y_0 \\
			 0, \quad x = x_1, y = y_1 \\
			 1, \quad x = x_2, y = y_2
		  \end{cases} \\
\end{cases}
$$

所以在此区间上 $\varphi_i = L_i$。

\section{形成线性方程组}

\subsection{刚度矩阵}

设 $\varphi_{xi} = (\varphi_i, 0)^t, \varphi_{yi} = (0,\varphi_i)^t$,i = 0, ... , n 为试探函数空间$U_h$的基函数,则任一 $u_h \in U_h$ 可表成
$$
	u_h = \sum\limits_{i=1}^n u^{i}_1 \varphi_{xi} + \sum\limits_{i=1}^n u^{i}_2 \varphi_{yi}, \quad u^i = u_h(x_i,y_i)
$$ 
\\ 
令 $\phi_{2i} = \varphi_{xi}$ ,$\phi_{2i+1} = \varphi_{yi}$ , $c_{2i} = u^i_1$ , $c_{2i+1} = u^i_2$
带入变分形式得
$$
	\sum\limits_{j=0}^{2n+1} a(\phi_j, \phi_i) c_i = (f,\phi_i) \quad i=0, ... ,2n+1
$$

矩阵形式为
$$
\begin{matrix}
	A c = F \\
	A = (a(\phi_i, \phi_j))_{(2n+1) \times (2n+1)} \\
	F = ((f,\phi_i))_{(2n+1) \times 1} \\
	c = (c_i)_{(2n+1) \times 1}
\end{matrix}
$$

%\subsection{单元刚度矩阵}
%
%在第m个单元cell=$\bigtriangleup(i,j,k)$上,单元刚度矩阵和单元载荷向量为
%$$
%\begin{matrix}
%	A^{(m)} = (\int_{cell} (2 \mu \epsilon(\phi_{i_1}) : \epsilon(\phi_{j_1}) + %\lambda div \phi_{i_1} div \phi_{j_1}))_{3 \times 3} \\
%	F^{(m)} = (\int_{cell} f \cdot \phi_{i_1})_{3 \times 1} \\
%	i_1, j_1 = i, j, k
%\end{matrix} 
%$$
%
%将$A^{(m)}$扩展成$n \times n$矩阵,行列为i,j,k的九个元素即为$A^{(m)}$的九个元素,并以同样的方式将$F^{(m)}$扩展成$n \times 1$向量,则
%$$
%\begin{matrix} 
%	A = \sum\limits_{m=0}^n A^{(m)} \\
%	F = \sum\limits_{m=0}^n F^{(m)}
%\end{matrix}
%$$

\subsection{边界条件}

模型为齐次边界条件,若$(x_i,y_i)$为边界点,则 A 第 2i 行第 2i 列,第 2i+1 行第 2i+1 列元素为1,其他元素及  F(2i),F(2i+1) 都为0。

%\vfill \newpage

%\bibliographystyle{unsrt}
%\bibliography{interface_problem}

\end{document}